\documentclass[12pt,english]{article} \usepackage{mathptmx} 
\usepackage{amsmath} \usepackage[top=1in, bottom=1in, left=1in, 
right=1in]{geometry} \usepackage[square,numbers]{natbib} 
\bibliographystyle{ksfh_nat} \setcitestyle{numbers,open={[},close={]}} 
\usepackage{url} \usepackage{booktabs} \usepackage{graphicx} 
\usepackage{pdflscape} \usepackage[table,xcdraw]{xcolor} 
\usepackage[unicode=true,pdfusetitle,
 bookmarks=true,bookmarksnumbered=false,bookmarksopen=false,
 breaklinks=true,pdfborder={0 0 0},backref=false,
 colorlinks,citecolor=black,filecolor=black,
 linkcolor=black,urlcolor=black]
 {hyperref} \linespread{2} \begin{document} \title{Price of a Life: An 
Application of Value-Adjusted Utility Theory to Pharmaceutical Pricing 
Policies} \author{Yaseen Mozaffar\thanks{Department of Economics, 
University of Oklahoma y.mozaffar@ou.edu}} \maketitle \vfill{} \pagebreak{} 
\section{Introduction}\label{sec:intro} At the core of consumer theory 
in economic consumer theory is utility, the subjective measure of 
satisfaction or usefulness gained through the consumption of a good or 
service. Utility's subjective nature is key, as it reflects the varying 
values and preferences of individual consumers and maximization of this 
value is the objective in rational consumer decision-making. Producer 
decisions, however, are not typically thought of as relying on utility 
maximization. The assumption is that firms operate without considering 
subjective preference, maximizing profit rather than utility. This paper 
applies Value-Adjusted Utility Theory, a particular theory that removes 
the distinction between profit-maximizing supply-side decisions and 
utility-maximizing consumer choice through the creation of a single 
utility function that can be used for any decision-maker. Through 
application of this theory to a case study of the pricing practices 
pharmaceutical companies under specific conditions, I demonstrate the 
process of supply-side decision-making through the VAU framework, as 
well as some potential uses of this theory to optimize decision-making 
over time. Following this introduction, the paper proceeds with a review 
of the literature surrounding Value-Adjusted Utility theory and relevant 
research on the pharmaceutical industry. The next section will detail 
the data used to select and analyze the case study, followed by a deeper 
analysis of the proposed utility function and its application to 
economic decision-making. The paper concludes with an analysis of the 
results of those methods, and the larger policy recommendations that 
come about as a result. \section{Literature Review}\label{sec:litreview} 
Utility is a widely studied concept within economics and various other 
disciplines, with nuanced distinctions made by different authors and 
different schools of thought. This paper stems largely from the 
conception of utility set out by Peter Fishburn in his 1968 survey of 
utility theories. He characterizes utility theories as sets of theorems 
derived from the axiomatic preference/indifference relationships between 
decision alternatives within a set \cite{fishburn68}. This paper centers 
around a utility theory that Fishburn would categorize as 
"prescriptive", or concerned with the normative applications of utility 
theory which draw conclusions about how agents ought to act. Though 
there are countless prescriptive utility theories, each with their own 
assumptions and interpretations, this paper applies the particular 
prescriptive theory advanced by Yaseen Mozaffar in his forthcoming 
paper, "Value-Adjusted Decision-Making: A Unified Utility Theory for 
Both Sides of a Market" \cite{mozaffar20}. As is typical for 
prescriptive utility theories, Value-Adjusted Utility Theory is a 
mechanism for rational economic decision-making, through maximization of 
a utility function. There are, however, some key departures from the 
conventions of utility theory. Most applicably to this paper, rather 
than a typical utility function, which is subject to an exogenous 
resource constraint, a VAU function recognizes that possession of 
resources has its own utility, so budgets are simply treated as another 
source of utility, as savings/profit or debt. In addition to providing a 
mechanism for valuations of utility more precise than an ordered list of 
preferences, this allows the utility function to be applied to producers 
through manipulation of a profit function. Another notable 
characteristic of VAU is that it categorizes all sources of utility as 
either resource adjustments (how a decision affects availability of 
time, money, or materials for the agent) or quality of life adjustments 
(how a decision affect overall welfare in the world). Illustration of 
the applications and implications of this theory on producer choice 
requires a subject that is as unrestrained by external factors as 
possible, to ensure that a firm's observed behavior is indeed a choice. 
In addition, for the sake of a comprehensible example, an ideal subject 
would have observable behavior that draws as direct a line between a 
quality of life adjustment and a resource adjustment. On both counts, 
the pharmaceutical industry is uniquely suited to serve as a case study. 
For one, many pharmaceutical companies operate as monopolies for certain 
products, which removes market forces that would typically alter the 
conditions of profit maximization. In her survey of the relationship 
between American intellectual property law and pharmaceutical companies, 
Lara Glasgow shows that pharmaceutical patents grant monopoly power that 
is functionally untouched by antitrust regulation, allowing them to 
price patent-protected products at any point, irrespective of market 
conditions\cite{glasgow_2001}. Importantly, she details a number of 
strategies employed and loopholes exploited by pharmaceutical 
manufacturers to extend the life of their patents long past the point 
where the costs of the product's research and development have been 
recouped\cite{glasgow_2001}. This strengthens the case that certain 
pharmaceutical companies can pursue elective pricing strategies, making 
the industry suitable for this analysis. Furthermore, prescription drugs 
provide an unusually clear relationship between producer resource 
adjustments and quality of life because consumption of pharmaceuticals 
are often the difference between life and death, not just some 
subjective measure of how satisfied someone is with their circumstances. 
This is reflected in a RAND review of empirical literature on the price 
elasticity of prescription drugs. Although different studies generated 
different results, each one found prescription drugs to be firmly 
inelastic, as low as -0.05 \cite{rand02}. Although the precise price 
elasticity is unknown for a particular drug or type of drug, the effects 
are not. Nicole Rende notes that the markets for particularly critical 
drugs may reach a state of "demand-supply inflexibility", wherein 
consumption is deemed so important to consumers that price elasticity is 
functionally 0, allowing virtually unrestrained price gouging 
\cite{rende_2019}. Coleen Cherici, Patrick McGinnis, & Wayne Russell 
further explore the relationship between price markups and how critical 
a drug is for those who need it. Their analysis found that among 416 
medicines used for "emergencies and serious conditions," were marked up 
an average 650\% from market pricing and that the drugs deemed "most 
critical" could be marked up as high as 4533\% \cite{pha11}. These 
percentages are not perfectly applicable to this paper because the 
particular study was focused on drug shortages caused by gray market 
supply chains, not monopolies, but nonetheless, the causal link between 
the necessity of a drug and the propensity to price gouge sufficiently 
illustrates the directness of the relationship between pharmaceutical 
resource adjustment and quality of life adjustments. Finally, a paper 
headed by Andrew Wilper found that lack of insurance could be causally 
linked to approximately 45,000 deaths per year among American adults 
under the age of 64 \cite{wilper09}. Again, while not directly about 
monopolistic pricing, this paper still firmly establishes a relationship 
between inability to pay for healthcare and resulting deaths. Taken 
together, this base of literature produces an inescapable conclusion 
that certain profit-maximizing pharmaceutical companies electively price 
patent-protected, life-saving drugs at levels that can be directly 
linked to some amount of deaths due to inability to pay. With that 
conclusion drawn, this paper will proceed to apply Value-Adjusted 
Utility Theory to pharmaceutical pricing, in order to explore the 
tradeoff being made between profitability and welfare. 
\section{Data}\label{sec:data} In order to properly apply VAU, a subject 
more precise than "certain pharmaceutical companies" is required. The 
process of isolating such a subject is detailed here. I began with the 
California Office of Health Planning and Development's dataset of 2019 
Prescription Drug Wholesale Acquisition Cost (WAC) 
Increases\cite{wac20}. This data originated in California because 
California is the only state to mandate reporting of this information, 
but every company that sells in California is required to report, so it 
is unlikely that any major drug manufacturers were excluded from this 
data. WAC is not reliably indicative of the price that consumers pay for 
a particular drug, so these data were far from sufficient, but it was 
nonetheless valuable as a starting point after extensive cleaning. 
Additionally, limiting the sample only to drugs that increased in price 
strengthened the degree to which a pricing decision could be attributed 
to company choice by removing any drug that may have been subject to 
some factor forcing that company to lower its prices (such as a demand 
shock or loss of overall profitability). I began by removing unnecessary 
variables from the dataset, leaving only information necessary for 
identifying the drug and company that produces it. I then filtered all 
but "single source drugs" (drugs only produced by a single company with 
no generic counterparts) out of the dataset, as well as any drug with a 
patent expiry date that has already passed, or was N/A (meaning there 
was not a patent on the drug). This restricted my sample to only 
patent-protected drugs over which a company held a monopoly. I then 
manipulated the NDC variable (a unique ID number reflecting the 
manufacturer, drug, and packaging of the drug) into a new ID variable 
which reflected only the drug ID without the other two elements. This 
allowed me to remove the duplicates of identical drugs that were 
packaged at different doses and quantities. Next, the average price paid 
by an American customer for a course of each medication was retrieved 
from GoodRX, along with the condition(s) the drug was meant to treat 
\cite{goodrx}. The dataset was then further restricted by comparing the 
target condition to the Lancet's data on mortality by disease, and 
removing any drug that treated only conditions that did not appear in 
the set (meaning that they are not meaningfully associated with 
mortality)\cite{cod18}. Also removed from the sample was any drug that 
only treated symptoms, rather than a particular disease, to avoid 
double-counting the effects of a drug that treated a disease that causes 
those symptoms. The final restriction on the dataset was to remove any 
drug made by a company with a negative net income, because a company 
that is not profitable cannot freely make decisions while remaining 
viable. The drugs remaining in the dataset now are unique, 
patent-protected, life-saving medications created by profitable 
companies. This is the closest approximation of purely elective 
decision-making that can be isolated with the available data. From here, 
a distinction needed to be made between "life-saving" and 
"life-extending" drugs; the majority of the drugs left in this dataset 
were used to mitigate the impact of a disease, but not actually cure it. 
While a metric of lifetime added could theoretically be calculated from 
such drugs, such a metric fails to create an intuitive enough 
relationship between profit and welfare to properly illustrate the 
tradeoff. The dataset has now been restricted just to five electively 
priced drugs which lower the probability of its consumer dying from a 
particular condition. With incidence data from the Centers for Disease 
Control and mortality differentials from clinical trials retrieved from 
the National Institutes of Health, I am able to examine the 
decision-making of these pharmaceutical companies through the lens of 
Value-Adjusted Utility\cite{cdc20}\cite{trials}. Table 1 displays all 
relevant information for these five drugs. \section{Empirical 
Methods}\label{sec:methods} The principles of Value-Adjusted Utility 
Theory were applied in two distinct ways to the case of elective 
pharmaceutical pricing decisions. Both stem from the general case of the 
Value-Adjusted utility function\cite{mozaffar20}: \begin{equation} U(x) 
= \sum {v_1\int\limits_0^x u_1x \ dx + v_2\int\limits_0^x u_2x \ dx ... 
+ v_n\int\limits_0^x u_nx \ dx }\end{equation} where \emph{U} is the 
utility received from taking action \emph{x} (if the decision is simply 
whether or not to act) or from taking \emph{x} units of action (if an 
action can have different effects based on the level of action), and 
each $u$ is the marginal utility of the a single outcome of the action 
either in terms of resources or quality of life, weighted by $v$, which 
is a numerical representation of the relative importance of that outcome 
for the agent, such that the set of values for v for the entire function 
sum to 1. Any decision made is represented as choosing the value of 
\emph{x} which maximizes U{x}. For the purpose of practical application 
to the pharmaceutical case, this function can be modeled using the 
following simplified form \cite{mozaffar20}: \begin{equation} U(x) = 
(v)u_{R+}x + (1-v)u_{QoL+}x- (v)u_{R-}x - (1-v)u_{QoL-}x\end{equation} 
In this case, we hold marginal utility constant and reduce the 
potentially infinite outcome terms to just the sum of the positive 
effects of the action minus the negative effects of the action. For 
simplicity in this case, we can refer to the resource terms as "money" 
and the quality of life terms as "lives" or "welfare." Therefore, the 
value adjustment reflects the relative importance of wealth accumulation 
and human life in the opinion of the agent. The decision in question is 
the price at which a firm elects to price a life-saving medication. 
Because firms like the pharmaceutical companies are profit-maximizing 
agents, we assume that they weight monetary terms with a value of 1, 
creating a utility function that is entirely defined as the difference 
between revenue gained and costs incurred. In other words, the 
Value-Adjusted Utility function of a purely profit-maximizing firm is 
identical to its profit function. Suppose now we reverse the weights on 
the utility function such that quality of life is weighted at 1 and 
profits become a non-factor. The pharmaceutical pricing strategies can 
be thought of as a discrete choice model between the two extremes of 
value adjustments. The quality of life-weighted counterfactual model 
displays the outcome if the firms were to price their product at \$0 so 
that anyone who needs the drug would be able to access it. The firm 
would then be forgoing a monetary gain equal to the number of people 
afflicted with the target condition multiplied by the average price of 
the drug. So long as this value is less than or equal to the net income 
of a firm, that firm could choose to price their product at \$0 without 
sacrificing anything other than shareholder profits.
 
\section{Research Findings}\label{sec:results} A typical discrete choice 
model seeks to estimate the probability of an actor selecting each 
option. In the case of these pharmaceutical companies, however, this is 
unnecessary; the option to maximize profit has already been selected in 
each case. The purpose of framing these pricing decisions as a matter of 
discrete choice is to demonstrate that the extreme values of 
value-adjustment (and therefore all of the values in between them) are 
viable strategies for these firms. Table 2 displays the outcome of the 
counterfactual model, in which pharmaceutical companies only consider 
quality of life in their pricing decisions, as well as some of the 
implications of its comparison with the status quo of profit 
maximization. The number of lives saved by this decision is calculated 
as the number of Americans afflicted with the target condition, 
multiplied by the difference in mortality between those treated with the 
drug in question and those treated with the next best option. Forgone 
Profits is the amount of money that a firm would lose if their drug was 
priced at \$0 instead of the Average Price and everyone afflicted 
"purchased" a course of the drug. Put another way, this value is the 
amount of profit gained by the firm's current price point rather than at 
\$0. For each of these drugs, the forgone profits were well below the 
threshold of the manufacturer's net income. Because net income is 
calculated as the revenue remaining after the value all expenses are 
deducted, these forgone profits would not affect the firm's research 
budget, or in any way reduce their ability to continue producing 
medication. There would simply be less money to be circulated among 
shareholders and executives in the form of dividends and bonuses. This 
is true even in the case of Novartis, as the sum of forgone profits for 
all three electively-priced drugs are below the net income threshold. By 
dividing the foregone profits by the number of lives that would be saved 
by making their drug freely available, therefore, we estimate how many 
dollars these firms have gained for each person they have chosen not to 
save from the target condition. The model has two uses. First, given 
reliable valuations, can be used to make decisions, just like the 
average utility function. Second, more importantly, can be used to 
reverse engineer valuations from decisions that have already been made. 
In the test case, we found that Novartis valued human life this much 
(insert table here). Test variance for statistical significance. 
Significance indicates that there is some factor making a difference in 
valuation outcomes between different drugs. 
\section{Conclusion}\label{sec:conclusion} The Value of a Life estimated 
in this paper is unlikely to be accurate. The mortality differential for 
target conditions was calculated using point estimates rather than 
confidence intervals, and they were drawn from limited sample sizes from 
clinical trials. Furthermore, while the heavily restricted selection of 
drugs under consideration approximated elective decision-making as 
closely as possible, there is no way to prove that those pricing 
decisions were in fact unaffected by market conditions or other factors, 
or that pricing pharmaceuticals at \$0 would not cause some quality of 
life tradeoff. Finally, of course, the Value-Adjusted Utility function 
was simplified to the point where even a complete data set would produce 
inaccurate results. Indeed, the extreme variation for the values of life 
calculated for the three Novartis entries implies that at least two of 
those estimates are incorrect estimations of the firm's value of life. 
None of this, however, is central to the purpose of this paper. The 
accuracy of the particular dollar value estimated as the price of a life 
is far less important than the fact that a value could be estimated in 
the first place. Under the typical assumptions of economic producer 
theory, firms are treated as value-neutral, objective decision-makers, 
for whom pure profit-maximization is not only an acceptable framework 
for action, but unquestioningly assumed to be "correct". Through the 
application of Value-Adjusted Utility to the case of pharmaceutical 
pricing and simple modeling techniques to demonstrate that profit 
maximization can be a choice rather than a necessity, I argue that 
profit maximization as a corporate objective is anything but 
value-neutral. Ultimately, decisions justified as purely "economic" (as 
are most corporate choices) entails the imposition and acceptance of a 
value system that treats societal welfare as a non-issue, sometimes at a 
cost of thousands of human lives each year. The widespread acceptance of 
pure profit maximization as the correct objective for a firm means that 
firms are never made to confront their effect on quality of life, except 
in cases where it can adversely affect profits, such as through legal 
action or negative press coverage. This theoretical utility framework 
illustrates that, irrespective of whether or not anyone acknowledges the 
quality of life cost of profit-maximization, that cost is incurred to a 
degree that is significant and measurable. The only policy 
recommendation that stems from this analysis is for economic 
decision-makers to interrogate their values, and in particular challenge 
the assumption that profit maximization is sufficient to absolve firms 
of the impacts of their actions. It is quite possible that even 
widespread re-imagining of such values would produce no tangible changes 
to the economy. A pharmaceutical executive may be confronted with the 
fact their pricing decisions have implicitly valued a human life at 
\$22,000 and not feel compelled to change, but at the very least they 
would lose the opportunity to pretend that they have never considered 
the human cost of their choices. It is also possible, however, that if 
annual reports were to include the number of people 
killed or harmed by the actions of a firm, giving the decision-makers at 
that firm no choice but to think about their impact, there might be more 
thought given to the inevitable tradeoff between pure profit-chasing and 
the cost it incurs. 
\vfill 
\pagebreak{} 
\clearpage \section*{References}
\bibliography{FinalProducts/references.bib} 
\addcontentsline{toc}{section}{References} 
\vfill 
\pagebreak{} 
\clearpage \section*{Figures and Tables} \begin{table}[htp] 
\resizebox{\textwidth}{!} & {\color[HTML]{000000} 6\%} \\ \midrule Eliquis & Bristol-Myers 
Squibb & 4.92 billion & AFib-related stroke & 2,200,000 & 
{\color[HTML]{000000} 15.00\%} & {\color[HTML]{000000} 9\%} \\ \midrule 
Entresto & Novartis & {\color[HTML]{0A0101} 12.611 billion} & Chronic 
Heart Failure & 6,500,000 & {\color[HTML]{000000} 1.11\%} & 
{\color[HTML]{000000} 0.90\%} \\ \midrule Tasigna & Novartis & 
{\color[HTML]{0A0101} 12.611 billion} & Chronic Myelogenous Leukemia & 
58,000 & {\color[HTML]{000000} 3.94\%} & {\color[HTML]{000000} 3.54\%} 
\\ \midrule Gilenya & Novartis & {\color[HTML]{0A0101} 12.611 billion} & 
MS Relapse & 250,000 & {\color[HTML]{000000} 20\%} & 
{\color[HTML]{000000} 17\%} \\ \bottomrule \end{tabular}%
}
\caption{Drug and Disease Info} \label{Table 1 } \end{table} 
\begin{table}[htp] \resizebox{\textwidth}{!}{% 
\begin{tabular}{|c|c|c|l|l|l|} \hline 
\multicolumn{1}{|l|}{\textbf{Drug}} & \multicolumn{1}{l|}{Company} & 
\multicolumn{1}{l|}{Net Income} & Forgone Profits & Lives Saved & 
\textbf{Value of a Life} \\ \hline Pradaxa & Boehringer Ingelheim & 9.9 
billion & {\color[HTML]{000000} \$1,101,362,000.00} & 4,615 & 
{\color[HTML]{000000} \textbf{\$238,648.30}} \\ \hline Eliquis & 
Bristol-Myers Squibb & 4.92 billion & {\color[HTML]{000000} 
\$3,366,500,000.00} & 8,800 & {\color[HTML]{000000} 
\textbf{\$382,556.80}} \\ \hline Entresto & Novartis & 
{\color[HTML]{0A0101} 12.611 billion} & {\color[HTML]{000000} 
\$1,185,800,000.00} & 16,250 & {\color[HTML]{000000} 
\textbf{\$72,972.31}} \\ \hline Tasigna & Novartis & 
{\color[HTML]{0A0101} 12.611 billion} & {\color[HTML]{000000} 
\$1,225,400,000.00} & 1,044 & {\color[HTML]{000000} 
\textbf{\$1,173,755.00}} \\ \hline Gilenya & Novartis & 
{\color[HTML]{0A0101} 12.611 billion} & {\color[HTML]{000000} 
\$4,303,000,000.00} & 195,000 & {\color[HTML]{000000} 
\textbf{\$22,066.67}} \\ \hline \end{tabular}%
}
\caption{Counterfactual Analysis} \label{table 2} \end{table}
\end{document}
